%23: 7, Munkres \textsection 24: 1, 3, 10e
% v0.04 by Eric J. Malm, 10 Mar 2005
\documentclass[12pt,letterpaper,boxed]{jhwhw}

\author{Sam Stewart}
\title{Functional Analysis}
% set 1-inch margins in the document
\usepackage{enumerate}
\usepackage{amsthm}
\usepackage{amsmath}
\usepackage{amssymb}
\usepackage{marginnote}
\usepackage{float}


\newtheorem{theorem}{Theorem}[section]
\newtheorem{lemma}[theorem]{Lemma}
\newtheorem{proposition}[theorem]{Proposition}
\newtheorem{corollary}[theorem]{Corollary}
\newtheorem{definition}{Definition}

\newcommand{\topologyn}{{\mathcal{T}}}
\newcommand{\indic}[1]{{\textbf{1}_{#1}}}
\newcommand{\topologyp}{{\mathcal{T}^\prime}}
\newcommand{\basis}{{\mathcal{B}}}
\newcommand{\topology}[1]{{\mathcal{T}_{#1}}}
\newcommand{\indicator}[1]{{\textbf{1}_{#1}}}
\newcommand{\Zplus}{{\mathbb{Z}_+}}
\newcommand{\Z}{{\mathbb{Z}}}
\newcommand{\Mf}{{\mathcal{M}_F}}
\newcommand{\M}{{\mathcal{M}}}
\newcommand{\eps}{{\varepsilon}}
\newcommand{\lebg}[1]{{\mu_L\left(#1\right)}}
\newcommand{\outerm}[1]{{\mu^*\left(#1\right)}}
\newcommand{\lebm}[1]{{\mu_L\left(#1\right)}}
\newcommand{\measure}[1]{{\mu\left(#1\right)}}
\newcommand{\ring}{\mathcal{R}}
\newcommand{\C}{{\mathbb{C}}}
\newcommand{\K}{{\mathbb{K}}}
\newcommand{\Nn}{{\mathbb{N}}}
\newcommand{\Rplus}{{\mathbb{R}_+}}
\newcommand{\rl}{{\mathbb{R}_l}}
\newcommand{\linfty}{l^\infty}
\newcommand{\closure}[1]{{\text{Cl}\left( #1 \right)}}
\newcommand{\Rmin}{{\mathbb{R}_-}}
\newcommand{\Q}{{\mathbb{Q}}}
\newcommand{\R}{{\mathbb{R}}}
\newcommand{\eR}{{\overline{\mathbb{R}}}}
\newcommand{\lebRing}{{\ring_{\text{Leb}}}}
\newcommand{\Rtwo}{{\mathbb{R}^2}}
\newcommand{\Rn}{{\mathbb{R}^n}}
\newcommand{\norm}[1]{\lVert #1 \rVert}
\newcommand{\dotp}[2]{{#1 \cdot #2}}
\newcommand{\abs}[1]{\left| #1 \right|}
\newcommand{\powerset}[1]{{\mathcal{P}(#1)}}
\newcommand{\puncplane}{{\ring^2 - \{ 0 \} }}
\newcommand{\puncplanen}{{\ring^n - \{ 0 \} }}
\newcommand{\til}[1]{{\widetilde{#1}}}
\newcommand{\degree}[2]{{deg_{#1} (#2)}}
\newcommand{\conj}[1]{{\overline{#1}}}
\newcommand{\series}[1]{{\sum_{k = 1}^\infty {#1}_k}}
\newcommand{\seq}[1]{{\left( {#1} \right)_{n = 1}^\infty }}
\newcommand{\maxm}[2]{{\max \, \left\{ {#1}, \, {#2} \right\} }}
\newcommand{\minm}[2]{{\min \, \left\{ {#1}, \, {#2} \right\} }}
\newcommand{\shortseq}[1]{{\left( {#1} \right) }}
\newcommand{\interior}[1]{{\textrm{Int} \left( {#1} \right)}}
\newcommand{\innerp}[2]{{\left< #1,\, #2\right>}}
% \newtheorem{lemma}[section]{Lemma}

\usepackage{graphicx}
\usepackage{float}

% Note: for other writers, please take a look at the shortcuts I have already defined above.

% TODO: employ roman numerals in the 
\begin{document}
\problem{Definitions of spectra for Linear Operators}
I include the definitions of the different pieces of the spectrum.

\begin{definition}[Spectrum]
	Let $X$ be a Banach space and $L$ a linear operator on $X$. Then the spectrum of $L$, denoted by $\Sigma\left( L \right)$, is the collection of $\lambda \in \C$, such that $(\lambda - L)$ does not have a bounded inverse in $X$.
\end{definition}

\begin{definition}[Resolvent Set]
	Let $X$ be a Banach space and $L$ a linear operator on $X$. The resolvent set $\rho(L) = \C - \Sigma\left( L \right)$. On this set, the operator $(\lambda - L)$ is bijective.
\end{definition}

\begin{definition}[Eigenvalue]
	We call $\lambda \in \C$ an eigenvalue if $(\lambda - L)$ is not injective.
\end{definition}

\begin{definition}[Fredholm Operator]
	An operator $L$ is Fredholm if the following are true
	\begin{enumerate}
		\item $\textrm{Ran } L$ is closed.
		\item $\textrm{codim Ran } L < \infty$
		\item $\textrm{dim Null } L < \infty$ 
	\end{enumerate}
\end{definition}

\begin{definition}[Fredholm Index]
	The Fredholm index of an operator $L$ is
	\[
		\textrm{dim Null } L - \textrm{codim Ran} L
	\]
\end{definition}

\begin{definition}[Point Spectrum]
	We say that $\lambda \in \Sigma_{pt}$ if $(\lambda - L)$ has Fredholm index zero.
\end{definition}

\begin{definition}[Essential Spectrum]
	We define the essential spectrum to be
	\[
		\sigma_{\textrm{ess}} = \Sigma - \Sigma_{\textrm{pt}}
	\]
\end{definition}

\begin{definition}[Approximate Point Spectrum]
	The approximate spectrum $\Sigma_{\textrm{app}}$ is the set of all $\lambda in \C$ such that $(\lambda - L)$ has a range which is not closed in $X$. 
\end{definition}

\begin{definition}[Residual Spectrum]
	The residual spectrum $\Sigma_{\textrm{res}}$ is the set of all $\lambda$ for which the range $(\lambda - L)$ is not dense in $X$.
\end{definition}

\end{document}
