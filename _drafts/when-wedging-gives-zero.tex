%23: 7, Munkres \textsection 24: 1, 3, 10e
% v0.04 by Eric J. Malm, 10 Mar 2005
\documentclass[12pt,letterpaper,boxed]{jhwhw}

\author{Sam Stewart}
\title{Differential Forms}
% set 1-inch margins in the document
\usepackage{enumerate}
\usepackage{amsthm}
\usepackage{amsmath}
\usepackage{amssymb}
\usepackage{marginnote}
\usepackage{float}


\newtheorem{theorem}{Theorem}[section]
\newtheorem{lemma}[theorem]{Lemma}
\newtheorem{proposition}[theorem]{Proposition}
\newtheorem{corollary}[theorem]{Corollary}

\newcommand{\topologyn}{{\mathcal{T}}}
\newcommand{\indic}[1]{{\textbf{1}_{#1}}}
\newcommand{\topologyp}{{\mathcal{T}^\prime}}
\newcommand{\basis}{{\mathcal{B}}}
\newcommand{\topology}[1]{{\mathcal{T}_{#1}}}
\newcommand{\indicator}[1]{{\textbf{1}_{#1}}}
\newcommand{\Zplus}{{\mathbb{Z}_+}}
\newcommand{\Z}{{\mathbb{Z}}}
\newcommand{\Mf}{{\mathcal{M}_F}}
\newcommand{\M}{{\mathcal{M}}}
\newcommand{\eps}{{\varepsilon}}
\newcommand{\lebg}[1]{{\mu_L\left(#1\right)}}
\newcommand{\outerm}[1]{{\mu^*\left(#1\right)}}
\newcommand{\lebm}[1]{{\mu_L\left(#1\right)}}
\newcommand{\measure}[1]{{\mu\left(#1\right)}}
\newcommand{\ring}{\mathcal{R}}
\newcommand{\C}{{\mathbb{C}}}
\newcommand{\K}{{\mathbb{K}}}
\newcommand{\Nn}{{\mathbb{N}}}
\newcommand{\Rplus}{{\mathbb{R}_+}}
\newcommand{\rl}{{\mathbb{R}_l}}
\newcommand{\linfty}{l^\infty}
\newcommand{\closure}[1]{{\text{Cl}\left( #1 \right)}}
\newcommand{\Rmin}{{\mathbb{R}_-}}
\newcommand{\Q}{{\mathbb{Q}}}
\newcommand{\R}{{\mathbb{R}}}
\newcommand{\eR}{{\overline{\mathbb{R}}}}
\newcommand{\lebRing}{{\ring_{\text{Leb}}}}
\newcommand{\Rtwo}{{\mathbb{R}^2}}
\newcommand{\Rn}{{\mathbb{R}^n}}
\newcommand{\norm}[1]{\lVert #1 \rVert}
\newcommand{\dotp}[2]{{#1 \cdot #2}}
\newcommand{\abs}[1]{\left| #1 \right|}
\newcommand{\powerset}[1]{{\mathcal{P}(#1)}}
\newcommand{\puncplane}{{\ring^2 - \{ 0 \} }}
\newcommand{\puncplanen}{{\ring^n - \{ 0 \} }}
\newcommand{\til}[1]{{\widetilde{#1}}}
\newcommand{\degree}[2]{{deg_{#1} (#2)}}
\newcommand{\conj}[1]{{\overline{#1}}}
\newcommand{\series}[1]{{\sum_{k = 1}^\infty {#1}_k}}
\newcommand{\seq}[1]{{\left( {#1} \right)_{n = 1}^\infty }}
\newcommand{\maxm}[2]{{\max \, \left\{ {#1}, \, {#2} \right\} }}
\newcommand{\minm}[2]{{\min \, \left\{ {#1}, \, {#2} \right\} }}
\newcommand{\shortseq}[1]{{\left( {#1} \right) }}
\newcommand{\interior}[1]{{\textrm{Int} \left( {#1} \right)}}
\newcommand{\innerp}[2]{{\left< #1,\, #2\right>}}
% \newtheorem{lemma}[section]{Lemma}

\usepackage{graphicx}
\usepackage{float}

% Note: for other writers, please take a look at the shortcuts I have already defined above.

% TODO: employ roman numerals in the 
\begin{document}
\problem{My Explanation of Covectors}

\problem{My Explanation of the Wedge Product}

\problem{9.2}
Show that covectors $\omega^1, \ldots, \omega^k$ on a finite-dimensional vector space are linearly dependent if and only if $\omega^1 \wedge \cdots \wedge \omega^n = 0$.

\solution
For the first direction, assume that $\omega^1, \ldots, \omega^k$ are linearly dependent. Let $X_1, \ldots, X_k \in V$. By Proposition 9.7e (p210), we have
\[
	\omega^1 \wedge \cdots \wedge \omega^k(X_1, \ldots, X_k) = \textrm{det}\left( \omega^i \left( X_j \right) \right).
\]
By definition, we have $\alpha_i \in \R$ such that at least two $\alpha_i \neq 0$ and $\alpha_i \omega^i(X) = 0$ for any $X$. This is equivalent to to the columns of $\omega^i \left( X_j \right)$ being linearly dependent. Thus, by properties of the determinant
\[
	\omega^1 \wedge \cdots \wedge \omega^k(X_1, \ldots, X_k) =  \textrm{det}\left( \omega^i \left( X_j \right) \right) = 0.
\]

For the other direction, assume that
\[
	\omega^1 \wedge \cdots \wedge \omega^k(X_1, \ldots, X_k) =  \textrm{det}\left( \omega^i \left( X_j \right) \right) = 0,
\]
Then the columns of $\omega^i \left( X_j \right)$ have columns that are linearly dependent. This is equivalent to $\omega^1, \ldots, \omega^k$ being linearly dependent.
\end{document}
