\documentclass[12pt]{article}
\usepackage[english]{babel}
\usepackage[utf8]{inputenc}
\usepackage{fancyhdr}
\usepackage{mathptmx}
\usepackage[letterpaper, portrait, margin=1in,headheight=60.pt]{geometry}

\title{}
\begin{document}

\fontfamily{ptm}
My back aching, I stood up from the field that I had been weeding for the past five hours. The heat of the French countryside was sweltering, and I was starting to question a  decision I had made five months ago in a chilly graduate office in Minnesota. Struggling to understand the farmer giving commands in rapid French and confused by the array of tools he handed me, I was more than outside my comfort zone. Had I made a mistake? Working on an organic farm was harder than I had anticipated. Maybe my excitement had clouded my better judgement. But an instant later, I remembered why I had traveled to France with nothing but a backpack and an interest in adventure. I was \textit{curious}.  

\textbf{Intellectual Merit}\\
My drive to explore began in high school when I left the comfort of my small hometown in Oregon, and moved alone to the Bay Area to spend the summer working at a new startup called SeatMe as an iPhone developer. I loved programming and competed in regional competitions, but I wanted to experience corporate life. A risk, the idea turned out to be fantastic, and I parlayed this first experience into another job during college, working nights, weekends and holidays at an advertising startup called PlayHaven in Portland, Oregon. I pitched myself as an Android and iPhone developer and slowly worked myself into a senior position on the developer team. 

I loved the work, but by my freshman summer, I wanted to try research. After giving a talk at the university based on my work in industry, I earned a letter of recommendation from one of the computer science faculty. This letter convinced Professor Drake to allow a freshman onto his REU project in machine learning. Our goal was to make our Computer Go player unbeatable. Computer Go is a model for many harder problems in machine learning. The work offered a completely different challenge from industry, and I found myself spending many evenings trying to optimize our program's performance in the hope of winning just a few more games. While pursuing one of my crazy optimizations, I implemented an algorithm that I found in a paper written by the Spanish computer scientist Dr.\thinspace Jacques Balsaldua. This led to a collaboration between myself, Professor Drake, and him that culminated in a publication in \textit{IEEE Transactions on AI in Games}. After that summer, I realized a career in research would give me the chance, unlike my previous work at startups, to produce entirely \textit{new} ideas.

But something else had changed during that first summer. Struggling to learn about Markov-Chain theory and apply it to our player, I was repeatedly frustrated by the dense mathematics. I knew these theoretical tools would dramatically improve our performance, but I simply did not have enough background to understand them. I realized that I was struggling with a language barrier. Having arrived at Lewis \& Clark College wanting to study computer science, I had suddenly discovered I needed a bigger toolset. To attack the ktypes of problems I wanted to solve, I needed dual training in computation and pure math. I needed to learn a new language.

So I changed my sophomore coursework. Instead of programming classes, I took  theoretical courses like mathematical statistics. The new language was a struggle. Programmers think differently than mathematicians: mathematicians are satisfied with existence, while programmers necessarily need constructions. I learned the material differently than the other math students. I would translate definitions and proofs into concrete computations. In the mathematical statistics class, I would simulate every theorem in R, before proving it. In this way, I gained personal intuition and slowly started to progress.

After a year of coursework, I was eager to try research again. I studied Orbigraphs, graph theoretic analogues of orbifolds, with Professor Liz Stanhope. While this project was truly pure math, I did not forget my computational roots. I wrote a library in Mathematica that checked our conjectures on thousands of randomly generated Orbigraphs in just a few minutes. This method revealed a surprising pattern. The Orbigraphs with a universal cover (an infinite tree), appeared to be precisely those that, when viewed as Markov-Chains, satisfied the so-called irreversibility property. I struggled for weeks to prove the connection before finally succeeding. When I finally saw the proof, I was thrilled. For me, unlike programming, where the details came together slowly but surely, the proof emerged suddenly, only after weeks of work. My work with Professor Drake convinced me to pursue research in general because I wanted to discover something new. My work with Professor Stanhope focused my interest to mathematical research because I loved the thrill of proving something new.

My excitment for trying something new did not stop when I arrived to the University of Minnesota. I wanted to combine my programming and math skills, but I didn't know how. So, I took a chance and enrolled in a year long course in fluid dynamics. I immediately loved the interplay between numerics, physics, and pure math. But I wanted to do some real research. Professor Sverak, now my advisor, was excited to work with a student who had computational skills. As described in my research statement, our combination of numerics and theory has produced some exciting insights into a one-dimensional model for the Euler equations.

My insatiable curiosity has pushed me to develop a formidable skillset in mathematics and programming that I will use to attack problems in fluid dynamics. I started with a passion for computer science, but realized the combination of math and programming was more powerful. The switch was not easy, but it has proved worthwhile. By the end of my first year of graduate work, I have accomplished what I wanted four years ago during my first summer of research: I have overcome the language barrier in mathematics. Having passed all four preliminary exams, I am well-equipped to begin research.

\textbf{Broader Impact}\\
My curiosity has also pushed me outside of math in three ways. First, I have engaged with minority communities in Minneapolis. Second, I have shared my passion for mathematics with the general public. Third, I am connected internationally. 

During my first semester, I volunteered for Teaching SMART, an after-school enrichment program that serves under-represented middle-school students. I spent my first few visits helping other graduate students who were leading the lessons, before organizing an activity of my own. I used EuclidTheGame, an online game, to teach the students about Euclidean constructions. Though the students were previously always restless, every student was quickly glued to the game, trying to build shapes of ever-increasing complexity. At the end of the session I asked the students a question that perplexed mathematicians for years: is it possible to construct a square with the same area as the unit circle? When I told them it was impossible, the students were shocked. One girl even offered to prove me wrong. These are the interactions that I love. 

Moving to the larger community, I have presented my work to the general public at a local theatre. Bryant Lake Bowl Theatre in Minneapolis hosts scientists for a talk, drinks,  and an intensive question session. Previously, the speakers have been only professors, and none of them mathematicians. Wanting to change this, I pitched several ideas to the coordinator before we settled on the question of regularity of the Navier-Stokes equations. The events are popular and thus attract an audience with varying background and age so I worked hard to prepare a talk that was approachable, entertaining, and exciting. Sprinkling jokes and anecdotes throughout the talk kept the audience interested and laughing, even through the technical parts. Judging from the questions at the end of the talk, audience members left the theatre with an appreciation of the core problem in fluid dynamics, and more importantly, with a better understanding of what mathematicians do all day.

Finally, beyond the community of Minneapolis, I am connected internationally. While a junior in college, I spent a semester abroad in the Budapest Semesters in Mathematics program during my junior year. Not only did I form a close network of peers, who are now studying mathematics in graduate programs around the world, I also participated in the inaugural language exchange program. Hungarian is a difficult language and my vocabulary was not exactly expansive, but I wanted to understand the Hungarian mindset, something that would have been impossible if I spoke only with the Americans in my cohort. My assigned language partner, a motherly policewoman who struggled with her English as I struggled with my Hungarian, gave me exactly this: a new perspective, and we became close friends. As I learned functional analysis, differential geometry, and Galois theory, I also learned that the United States is not the center of the universe, and nothing is more simultaneously humbling and exhilarating than learning a new language. The entire experience, mathematically and culturally, expanded my horizens tremendously. 

My curiosity nudges me to explore, to step outside my comfort zone. It's a quiet restlessness in the back of my mind. But my second strength is focus. It is a counteractive force to my curiosity. Once I've set a goal for myself, I finish it. From the Portland marathon, to passing all four preliminary examinations in my first year, this focus keeps me on track. My path has been a feedback loop of these two traits: curiosity keeps me excited while focus keeps me moving. 
\end{document}
