\documentclass[12pt]{article}
\usepackage[english]{babel}
\usepackage[utf8]{inputenc}
\usepackage{fancyhdr}
\usepackage{mathptmx}
\usepackage[letterpaper, portrait, margin=1in,headheight=60.pt]{geometry}

\title{}
\begin{document}

\fontfamily{ptm}
My back aching, I stood up from the field I had been weeding in for the past five hours and took a break. The heat of the French countryside was sweltering and I was starting to question a  decision I had made five months ago and many miles away in a chilly graduate office in Minnesota. Struggling to understand the farmer giving commands in rapid French and confused by the array of tools he handed me, I was more than outside my comfort zone. Had I made a foolish mistake by deciding to pursue my interest in the French language by spending my summer vacation working on a small organic farm in the French countryside? Maybe my excitement had clouded my better judgement. But an instant later, I remembered why I had left Minnesota after my first year of graduate school with a backpack and an interest in adventure. I was \textit{curious}.  

My curiosity stretches back to high school where I decided it was a good idea (and managed to convince my parents that it was a good idea) to move alone to the bay area to spend the summer working at a new startup called SeatMe as an iPhone developer. The idea turned out to be fantastic, and I parlayed this into several other jobs during college, working nights, weekends and holidays at an advertising startup called PlayHaven in Portland, Oregon. I pitched myself as an Android and iPhone developer and slowly worked myself into a senior position on the developer team. By the end of college, I was mentoring new hires who were several years older than me.

Having already worked at several startups, I was curious about research after my freshman year of college. I joined Professor Drake's REU and worked on machine learning in the context of the game Go. Despite being the only freshman on the team, I implemented an algorithm that started a collaboration between Professor Drake and Dr. Jacques Balsaldua (?) and then to a co-authored publication in \textit{IEEE Transactions on AI in Games}. Most importantly for me, the advanced statistics sparked my curiosity in pure math. 

Pursuing my newfound interest in pure math, I spent my second summer with Professor Liz Stanhope studying Orbigraphs, a graph theoretic analogue of orbifolds. The process of discovering a surprising link between Markov chains and the universal covering trees of Orbigraphs, then presenting my work at the 2013 Young Mathematicians' Conference, convinced me to pursue math at the graduate level. 

I knew that I needed more background before applying to competitive programs, so I traveled to Hungary to spend a semester in the renowned Budapest Semesters in Mathematics Program. Mathematically and culturally, the experience was amazing and prepared me for my senior thesis.

Knowing that wanted to pursue math, my curiosity pushed me to try something in analysis since I had spent the previous summer studying algebra with Professor Stanhope. I worked with Professor Allen for my senior thesis learning about PDEs and numerical methods, and then presented our results as a talk at the 2015 Joint Mathematical Meetings. By the end of the year, the  curiosity that had started in high school had narrowed to numerical analysis and PDEs. I realized that I could utilize my dual skills in programming and math. 

Interested in PDEs, my curiosity guided me to take a year long course in fluid dynamics during my first semester of graduate school. Most first-year students take two or three core courses, by having passed the real analysis exam immediately upon entering, I decided to take a risk. This chance blossomed into a fascination with fluids and a publication in-preparation with my adviser Professor Sverak. Fluids fascinate me for two main reasons. Firstly, Professor Sverak approaches fluids from a geometric perspective that complements the visualizations I can produce with my programming skills. Secondly, the physical motivation. Studying fluids has dramatically changed how I perceive the world. I am shocked that apparently simple substances like air and water hide such complexity that strain our mathematical tools. 

% curiosity drives my broader impact
Also since high school, my curiosity has encouraged me to explore outside of mathematics. I love mathematics, but I also love interacting with the world outside of academia.

While in Hungary, my curiosity convinced me to participate in the inaugural language exchange program. Hungarian is a difficult language and my vocabulary was not exactly expansive, but I really wanted to understand the Hungarian mindset, something that would have been impossible if I spoke only with the Americans in my cohort. My assigned language partner, a motherly policewoman who struggled with her English as I struggled with my Hungarian, gave me exactly this: a new perspective, and we became fast-friends. As I learned functional analysis, differential geometry, and Galois theory, I also learned that the United States is not the center of the universe, and nothing is more simultaneously humbling and exhilarating than learning a new language. Like mathematics, a new language is more than peculiar syntax and grammar, and is instead a new way of \textit{thinking}.  

Even while immersed in my first year of graduate school,  curiosity convinced me to step outside my department. I cold-emailed, the Council of Graduate Students, and became the math department's representative, a role which allowed me to interact with graduate students across the university. I quickly earned the assignment of organizing the entire set  of events for the Minneapolis campus during Graduate Student Appreciation Week. This pushed me to develop new organization skills and expanded my social network dramatically. Through these events I met graduate students from almost every department, many of whom have become close friends. 

Beyond other departments, my curiosity pushed me to explore the local community of Minneapolis. I volunteered for Teaching SMART, a program that serves under-represented middle-school students at a charter school near the university. I helped a few times to learn the system, and then asked to organize an activity of my own. A topic not usually covered until high school, I used an online game to teach the students about Euclidean constructions. Though many of the students had behavioral problems and were often restless, every student was quickly glued to the game, trying to build shapes of ever-increasing complexity. At the end of the session I asked the students if they thought it was possible to construct a square with the same area as the unit circle. When I told them it was impossible, the students were shocked. One girl offered to prove me wrong. It are these interactions that keep me motivated to share my passion for math with the local community. 

% One of the events I hosted was a night at our local Cafe Scientifique, a popular science presentation at a local theatre. Surrounded by graduate students from across the university, I listened to a history professor from the University of Utah describe his research trip into Tibet. 

\textbf{Intellectual Merit}\\
In addition to my programming and undergraduate background in mathematics, I  have cleared all four required preliminary exams during  my first year. I didn't take any of the associated courses: I self-studied and organized study groups for each topic. This independence taught me how to stay motivated, disciplined, and how to absorb new math quickly. The preliminary exams are necessary, but are not sufficient for research. My first year, I also took a year long course in fluids and PDE theory and am co-authoring a paper with Professor Sverak. Having learned the basics of harmonic analysis and spectral methods during  my project with Professor Sverak, I plan to delve deeper by taking a harmonic analysis course and a numerical methods for PDE course next semester.

\textbf{Broader Impact}\\
Since arriving at the university, I have made an effort to engage with other departments, the local community, and even internationally. My work with COGS places me in contact with students from many different departments. For example, as I work with a team of psychology students to deploy a mental- and physical-health survey in the math department, I am regularly humbled by the expanse of knowledge.

In the local community, I worked with Teaching SMART, gave a talk at Minneapolis Community College about the career paths available to mathematicians,  tutored a student with cerebral palsy in an advanced PDE course, and volunteered at an emergency food shelter. All of these activities were tremendously humbling and dramatically expanded my perspective.

Personally, my interest in French has enabled me to learn from a completely new set of people. My weekly attendance at a local French group has won me several new friends whose international perspectives I value tremendously and which prepared me for my travels in France this summer. While working on the farm itself, my discussions with my hosts about science, capitalism, and modern agriculture changed many of my naive assumptions. 

I also gravitate to international students as my friend group. In Minnesota, I befriended two visiting Spanish PhD students and keep in contact with them. My closest friends are an Indian and Turk. Even at conferences, I find myself drawn to graduate students with different backgrounds. Last summer I attended the Chicago Summer School in Analysis and befriended a Bosnian and Iranian and we have stayed in touch. This continued this May when I attended a Columbia summer school in variational methods for PDEs, I became close friends with a student at Paris XI and a southern Indian student. I visited the French student while traveling in France, and the Indian and I Skype weekly as we work our way through spectral theory.

Curiosity is one of my two core strengths. It is what nudges me to explore, to step outside my comfort zone. It's a quiet restlessness in the back of my mind. But my second strength is focus. It is a counteractive force to my curiosity. Once I've set a goal for myself, I finish it. From the Portland marathon, to passing all four preliminary examinations in my first year, this focus keeps me on track. My path has been a feedback loop of these two traits: curiosity keeps me excited while focus keeps me moving. 

I'm confident that the University of Minnesota is the perfect place to pursue my excitement for fluid dynamics. Since fluids span physics, numerical analysis, and pure math, the presence of the Institute for Applied Mathematics and the tight integration between the pure and applied math departments is crucial. Professor Sverak and I work well together. His geometric perspective on fluids and his uncanny physical intuition complements my programming skill set. Working with him enables me to internalize the theory while progressing in research. I am thrilled to pursue my excitement for fluid dynamics in such an environment.
\end{document}
