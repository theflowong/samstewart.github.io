\documentclass[12pt]{article}
\usepackage[english]{babel}
\usepackage[utf8]{inputenc}
\usepackage{fancyhdr}
\usepackage{mathptmx}
\usepackage[letterpaper, portrait, margin=1in,headheight=60.pt]{geometry}

\title{}
\begin{document}

\fontfamily{ptm}
My back aching, I stood up from the field that I had been weeding for the past five hours. The heat of the French countryside was sweltering and I was starting to question a  decision I had made five months ago in a chilly graduate office in Minnesota. Struggling to understand the farmer giving commands in rapid French and confused by the array of tools he handed me, I was more than outside my comfort zone. Had I made a mistake? Working on an organic farm was harder than I had anticipated. Maybe my excitement had clouded my better judgement. But an instant later, I remembered why I had traveled to France with nothing but a backpack and an interest in adventure. I was \textit{curious}.  

My curiosity stretches back to high school when I left the comfort of my small hometown in Oregon, and moved alone to the Bay Area to spend the summer working at a new startup called SeatMe as an iPhone developer. I loved programming and competed in regional competitions, but I wanted to experience corporate life. A risk, the idea turned out to be fantastic, and I parlayed this first experience into another job during college, working nights, weekends and holidays at an advertising startup called PlayHaven in Portland, Oregon. I pitched myself as an Android and iPhone developer and slowly worked myself into a senior position on the developer team. 

I loved the work, but by my freshman summer, I wanted to try something new. Talking to professors, research sounded so different from the corporate world. After giving a talk at the university based on my work in industry, I earned a letter of recommendation from one of the computer science faculty. This letter convinced Professor Drake to allow a freshman onto his REU project in machine learning. We to improve our computer Go player. Computer Go is a model for many harder problems in machine learning. The work offered a completely different challenge from industry, and I found myself spending many evenings trying to optimize our program's performance in the hope of winning just a few more games. While pursuing one of my crazy optimizations, I implemented an algorithm that I found in a paper written by a Spanish researcher named Dr. Jacques Balsaldua. This led to a collaboration between him and Professor Drake that culminated in a co-authored publication in \textit{IEEE Transactions on AI in Games}. After that summer, I realized a career in research would give me the chance, unlike my previous work at startups, to produce entirely \textit{new} ideas.

But something had happened during that summer that changed my college trajectory. I had been struggling to learn about Markov-Chain theory with the goal of applying it to our player, but was repeatedly frustrated at the dense mathematics. I knew these theoretical tools would dramatically improve our performance, but I simply didn't have enough background to understand them. By the end of the summer, I realized that I was struggling with a language barrier. Having arrived at Lewis \& Clark College wanting to study computer science, I suddenly realized it wouldn't be enough. To solve the problems that I wanted to, I needed dual training in computation and pure math. I needed to learn a new language.

So I changed my sophomore coursework. Instead of applied programming classes, I took more theoretical courses like mathematical statistics. The new language was a struggle. Programmers think differently than mathematicians, but I was curious about this new way of thinking. Mathematicians are often satisfied with existence, while programmers necessarily need constructions. I learned the material differently than the other math students. I would translate definitions and proofs into concrete computations. In the mathematical statistics class, I would simulate every theorem in R, before proving it. In this way, I gained personal intuition and slowly started to progress.

After a year of rigorous math coursework, I was eager to try research again. I studied Orbigraphs, graph theoretic analogues of orbifolds, with Professor Liz Stanhope. While this project was truly pure math, I didn't forget my computational roots. I wrote a library in Mathematica that checked our conjectures on thousands of randomly generated Orbigraphs in just a few minutes. This process produced a surprising pattern. The Orbigraphs with a universal cover (an infinite tree), appeared to be precisely those that, when viewed as Markov-Chains, satisfied the so-called irreversibility property. I struggled for weeks to prove the connection before finally succeeding. When I finally saw the proof, I was thrilled. For me, unlike programming, where the details came together slowly but surely, the proof emerged suddenly, only after weeks of work. My work with Professor Drake convinced me to pursue research because I loved the challenging problem. My work with Professor Stanhope convinced me to pursue research in math because I loved the thrill of proving something new.

My excitment for trying something new did not stop when I arrived to the University of Minnesota. I wanted to combine my programming and math skills, but I didn't know how. So, I took a chance and enrolled in a year long course in fluid dynamics. I knew nothing about fluids, but over the span of a year, I grew to respect their beauty and complexity. Not content to solve homework problems, I wanted to do some real research. Professor Sverak, now my advisor, was excited to work with a student who had computational skills. As described in my research statement, our combination of numerics and theory has produced some exciting insights into a one-dimensional model for the Euler equations.

I am confident that the University of Minnesota is the perfect place to follow my interest in fluid dynamics. Professor Sverak's knowledge and experience in fluids is extensive, and more importantly,  we work well together. His geometric perspective and uncanny physical intuition complement my programming skills. I am thrilled to pursue my excitement for fluid dynamics in such an environment.

My journey to graduate school in math has had many turns, due to my insatiable curiosity. I started with a passion for computer science, but realized the combination of math and programming was more powerful. The switch was not easy, but it has proved worthwhile. By the end of my first year of graduate work, I have accomplished what I wanted four years ago during my first summer of research: I have overcome the language barrier in mathematics. But my personal journey has not happened only in mathematics. My ever-present curiosity has also pushed me to interact with the world outside of of math.

Wanting to learn about a new culture, I spent a semester abroad in Hungary. Not content to study only math, I participated in the inaugural language exchange program. Hungarian is a difficult language and my vocabulary was not exactly expansive, but I really wanted to understand the Hungarian mindset, something that would have been impossible if I spoke only with the Americans in my cohort. My assigned language partner, a motherly policewoman who struggled with her English as I struggled with my Hungarian, gave me exactly this: a new perspective, and we became close friends. As I learned functional analysis, differential geometry, and Galois theory, I also learned that the United States is not the center of the universe, and nothing is more simultaneously humbling and exhilarating than learning a new language. Like mathematics, a new language is more than peculiar syntax and grammar, and is instead a new way of \textit{thinking}.  

My first year of graduate school followed the same pattern. Even while swamped with coursework, my curiosity pushed me outside my department. I became the math department's representative to the Council of Graduate Students, a role that allows me to interact with graduate students across the university. I quickly earned the assignment of organizing the entire set  of social events for the Minneapolis campus. This pushed me to develop new organization skills and expanded my social network dramatically. 

Stepping outside of the university, I have also explored the local community of Minneapolis via service. I volunteered for Teaching SMART, a program that serves under-represented middle-school students at a charter school near the university. I helped a few times to learn the system, and then asked to organize an activity of my own. A topic not usually covered until high school, I used an online game to teach the students about Euclidean constructions. Though many of the students had behavioral problems and were often restless, every student was quickly glued to the game, trying to build shapes of ever-increasing complexity. At the end of the session I asked the students a question that perplexed mathematicians for years: is it possible to construct a square with the same area as the unit circle? When I told them it was impossible, the students were shocked. One girl even offered to prove me wrong. I loved learning about the students, and sharing my excitement for math with them.


\textbf{Intellectual Merit}\\
Usually finished by the second year, I have cleared all four required preliminary exams during my first year despite not having taken any of the associated courses. The only first-year in the class, I have also taken a year-long PDE sequence and am now prepared to attack research-level problems. 

\textbf{Broader Impact}\\
In addition to the outreach mentioned above, I have tutored a student with cerebral palsy in an advanced PDE course and volunteered at an emergency food shelter. Both of these activities were tremendously humbling and dramatically expanded my perspective.

My experiences abroad and at conferences have enabled me to build an international network of friends that have continued to shape my perspective, both mathematically and personally. For example, I am learning spectral theory with an Indian friend I met while in New York. I am learning math, but I am also learning how a different culture thinks. I had a similar experience in France. After working on the farm, I connected with a friend from another conference. We discussed math, of course, but I also had the chance to experience life in Paris. My excitement for meeting new people has given me many new perspectives.

Curiosity is one of my two core strengths. It is what nudges me to explore, to step outside my comfort zone. It's a quiet restlessness in the back of my mind. But my second strength is focus. It is a counteractive force to my curiosity. Once I've set a goal for myself, I finish it. From the Portland marathon, to passing all four preliminary examinations in my first year, this focus keeps me on track. My path has been a feedback loop of these two traits: curiosity keeps me excited while focus keeps me moving. 

\end{document}
