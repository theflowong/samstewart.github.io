\documentclass[12pt]{article}
\usepackage[english]{babel}
\usepackage[utf8]{inputenc}
\usepackage{fancyhdr}
\usepackage{mathptmx}
\usepackage[letterpaper, portrait, margin=1in,headheight=0pt]{geometry}
\title{}
\begin{document}

\fontfamily{ptm}
As I stepped off the BART train into the San Francisco fog, a stream of suits jostled me. Gaping at the skyscrapers, I worried that I had made a mistake. I had just turned seventeen, and it was my first day of work at a tech startup. I ducked into a Starbucks and watched the pedestrians hurry by. Slowly, I remembered what had pushed me to cold-call the startup three months ago: I wanted to improve my programming skills, learn how to work on a team, and live on my own in San Francisco. Remembering all this, I stood up  and walked back into the stream of people. I passed through the revolving doors of 220 Montgomery Street. I hurried up several flights of stairs, and knocked on the huge double doors: my adventure was about to begin. As the summer months passed, I became fascinated by the work. I stayed late and arrived early. I traveled to nearby San Jose and competed in hackathons. I visited conferences and networked with CEOs. By the end of the summer I was completely enveloped by startup culture.

\textbf{Intellectual Merit.} My experience in San Francisco is one example of a larger trend in my journey to graduate school in mathematics: first, I try something new with the goal of expanding my perspective and skill set, then I focus and maximize the opportunity. \textbf{Two such ventures have influenced my path to graduate school: working as a contract programmer and switching my focus from computer science to math.}

When the internet startup offered me a full-time position if I did not attend college, leaving became a hard decision. As a seventeen-year-old immersed in startup culture, the work provided me purpose, an amazing salary, and praise. College was unknown and the pay certainly was not as good. However, I realized that it was too early to settle. I needed to explore college before choosing my path.

Arriving at Lewis \& Clark College, I loved programming and planned to work in industry as soon as I graduated. While the college schedule was demanding, \textbf{I decided to try working as a contract software developer}. I wanted to continue to hone my programming skills outside of the classroom. Finding work was challenging, especially since I was so young, but I persisted. Eventually, I convinced PlayHaven, a local advertising startup, that I could port their entire iPhone platform to Android. \textbf{I threw myself into the project, working nights and weekends on their codebase to complete the work under their deadline of a month}. At school, I wrote papers on the French revolution, while at PlayHaven I stayed late (even when the heating stopped!) to make my deadline.

Not content to develop only my skillset in industry, I simultaneously worked to develop my skillset in research. I gave a talk on my work at Lewis \& Clark College and impressed Professor Drake, a computer science professor with a research experience for undergraduates (REU) in machine learning. During my freshman summer, he invited me onto his team. Our goal was to improve the performance of our computer Go player (a model problem in artificial intelligence akin to Chess). At the time, computer Go was ``unsolved'' in the sense that computers were unable to beat the best human players. \textbf{I spent nights trying to optimize our program's performance in the hope of making a smarter AI}. At the end of the summer, I had co-authored a paper on a statistical compression algorithm we had implemented, and I had presented a poster on a supervised learning project. My research and programming skills had improved tremendously and I was excited to continue the research in the spring. At the startup, my work had also won recognition. As with the first company, the CTO offered me a full-time position if I dropped out of college. I again took a few weeks to reflect, before politely declining the offer. I would still work as a contract programmer for the remainder of my college degree, but \textbf{my interest had shifted in two ways while working with Professor Drake}. First, I loved the thrill of doing something new in research. My contract programming work was exciting, but I enjoyed the challenge of inventing novel techniques in research. Second, I became obsessed with math. Trying to improve the performance of our computer Go player, I had started talking with Professor Chen, a professor of mathematical statistics, about using Markov chains for our project. He lacked the programming background to implement the techniques, while I lacked the theoretical background to understand the math. As I struggled through the dense papers, I realized that I needed to expand my skillset again. The problems I wanted to solve lay at the intersection of computation and theory. I had practical programming knowledge from industry, but to solve the problems I was interested in I would need to complement my programming background with a theoretical perspective.

\textbf{I dropped my planned sophomore computer science courses and took a year-long mathematical statistics course with Professor Chen}. This switch from  programming to theory was difficult. After struggling for weeks, I found a way to use my programming skills to improve my mathematical understanding: I began to simulate every theorem numerically before trying to prove it. I would simulate different scenarios and play with edge cases. In this way, I leveraged my programming skills to gain mathematical intuition.

I wanted to expand my mathematical tools, so applied to a pure math summer project at Lewis \& Clark College. Led by Professor Stanhope, a geometer, our goal was to characterize the orbigraphs (graph theoretic analogues of orbifolds) that had universal covers (infinite trees). I was fascinated by the project and spent evenings poring over introductory abstract algebra textbooks. Once I learned enough to understood the problem, I realized that I could utilize my computational background. I wrote a Mathematica library for testing our conjectures on thousands of randomly generated graphs. \textbf{This novel approach revealed a surprising pattern: the orbigraphs with universal covers were precisely those that, when viewed as Markov chains, satisfied the so-called irreversibility property}, providing a tantalizing glimpse of a relationship between the geometry of these objects and a seemingly unrelated probabilistic concept. However, this was just a (numerically supported) conjecture, I still needed to prove it. After weeks of struggle, I was thrilled to prove the connection. Still, for weeks after the project, I wondered why there should even be some relation between such disparate concepts. This hidden structure, only revealed with the right perspective, was fascinating to me. This project taught me that the computational skills I learned in industry could be used to discover patterns and conjectures, while a mathematical perspective could explain the mechanisms behind these patterns. Wanting to combine my two hard-won skill sets, \textbf{I discovered the perfect mix in numerical analysis and partial differential equations during my senior thesis with Professor Allen}. We wanted to construct blowup solutions to a nonlinear wave equation motivated by black hole formation in a massless scalar field. I used finite differences, Python and NumPy, Fortran, and deep results from the theory of partial differential equations to discover numerical evidence (which I presented at the 2015 Joint Mathematical Meetings) that our norm was too weak to predict blowup. I had found the field I wanted to pursue in graduate school.

The University of Minnesota was my first choice given its outstanding faculty in both numerical analysis and PDE. I was excited to work under Professor Sverak, whose willingness to use computation as a tool to attack theoretical problems in fluid dynamics meshed perfectly with my own interests. In the course of our collaboration over the past year, I have already needed to learn tools in functional analysis, PDEs, harmonic analysis, and Lie theory, while I have simultaneously expanded my computational background in spectral methods. Professor Sverak has taught me how to reduce complicated theories to core concepts, and how to find the relevant details of a problem. During our meetings, I test our conjectures numerically while Professor Sverak fields my many theoretical questions. Through this collaborative cycle, we first discovered numerical evidence of linearly stable equilibrium solutions to a 1D fluid model and then used spectral theory, Fourier analysis, and complex ODE theory to prove the behavior. We are preparing these results for publication. Using this project as a starting point, I plan to continue to build my background in mathematics and programming with the goal of attacking problems in fluid dynamics that lie at the intersection of numerical analysis and partial differential equations. My dual skillset will allow me to both make progress in two simultaneous directions and collaborate with researchers in both disciplines.

\textbf{Broader Impact.} I have interacted with the broader community at three different levels. First, I have worked to make math accessible to underrepresented and differently-abled students within the university. Second, I have shared my excitement for math with minority students in the local community. Finally, I have presented my passion for research with the general public.

My first broader impact is at the university level. I arrived at the University of Minnesota with an excitement for PDEs, and listed myself as a tutor specializing in the subject. I received a tutoring request for an advanced undergraduate PDE course from a student with cerebral palsy who used a power chair. Over the course of a semester, I taught Deandra PDEs, but I did most of the learning.  She taught me not to ignore those who communicate differently, to listen more and speak less, and to have the tenacity to pursue one's goals. Deandra changed my perspective -- both on mathematics and society -- and I truly enjoyed working with her.

My second broader impact is at the community level. Last fall, I volunteered for Teaching SMART, \textbf{an after-school enrichment program that serves underrepresented middle-school students}. I helped other graduate students lead lessons, and then organized an activity of my own. I used EuclidTheGame, an online game, to teach the students about Euclidean constructions. Every student was quickly glued to the challenge, trying to build shapes of ever-increasing complexity. At the end of the session I asked the students a question that perplexed mathematicians for years: given the Euclidean axioms, is it possible to construct a square with the same area as the unit circle? When I told them it was impossible, the students were shocked. One girl even offered to prove me wrong. These are the interactions that I love. 

Finally, as mentioned in the research statement, I will impact the general public when I give a presentation about the problem of turbulence at Minneapolis's local chapter of Cafe Scientifique on January 19th. An \textbf{international public outreach program which sells out weeks in advance, scientists give an accessible talk on their work and then take questions}. Previously, the speakers have been mostly tenured professors, and none of them mathematicians. Wanting to fill this void, I managed to convince the coordinator that math is relevant and fascinating.  She gave me a spot, and even invited me onto a local radio program to talk about math. My goal is for audience members to leave the cafe with a better understanding of the core problem in fluid dynamics, and more importantly, with a better understanding of what mathematicians do all day.

Over the past five years, like my first day in San Francisco, I have stood in front of a door, unsure of what lay beyond, and still knocked. When the door has opened, I have pushed hard to achieve my goals. My late nights started in a cramped startup office in San Francisco, and have continued through research as I have found myself repeatedly hooked on a challenging problem. My willingness to explore new perspectives combined with my ability to focus has enabled me to develop a formidable skill set in mathematics and computation that I will use to solve key problems in fluid dynamics.

\end{document}
