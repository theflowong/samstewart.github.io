\documentclass[12pt]{article}
\usepackage[english]{babel}
\usepackage[utf8]{inputenc}
\usepackage{url}
\usepackage{fancyhdr}
\usepackage{mathptmx}
\usepackage{amsmath}
\usepackage[letterpaper, portrait, margin=1in,headheight=0pt, headsep=0pt]{geometry}
\addtolength{\topmargin}{-20pt}

\title{}
\begin{document}

\fontfamily{ptm}

\begin{center}
    \textit{A 1D model for Inviscid Fluid Mechanics}
\end{center}

\textbf{Intellectual Merit.} Blowup of solutions from smooth initial data for the Euler and Navier-Stokes  equations is a fundamental open question in fluid dynamics.  A blow-up of Navier-Stokes would  be evidence that canonical models might be insufficient, whereas blow-up of Euler would be relevant for some of the basic ideas in turbulence theory. These are of course notoriously hard problems.

One potential approach, which dates at least back to Burgers \cite{sverak2014}, is  to obtain insights by studying singularity formation or regularity for 1D model equations. One designs these models to mirror important features of the real flows. Professor Sverak, Professor Jia (currently at the IAS in Princeton), and I are studying one such model, with plans for a later extension of our studies to dimensions two and three.

The model we study, proposed by De Gregorio in 1990,  is related to several other 1D models. The first is the Constantin-Lax-Majda model  \cite{constantin1985}. Written with $\omega$ as vorticity and $u$ as velocity, the model is
\begin{equation}
	\omega_t = u_x \omega,
	\label{eq:constantin}
\end{equation}
with the Biot-Savart law $u_x = H \omega$ where $H$ is the Hilbert transform. The authors exhibited many solutions that blowup in finite time. 

The second model, a 1D analogy to the 2D Boussinesq system in the half-plane (itself a model of axi-symmetric 3D Euler), is given by
\begin{equation}
	\begin{aligned}
		\omega_t + u \omega_x &= \theta_x \cr
		\theta_t + u \theta_x &= 0
	\end{aligned}
	\label{eq:boussinesq_system}
\end{equation}
with the same Biot-Savart law $u_x = H \omega$ as for (\ref{eq:constantin}), and an additional variable $\theta$ representing temperature. The authors in  \cite{sverak2014} gave a nonconstructive proof of blowup.

The De Gregorio model is similar to  (\ref{eq:constantin}) except for the addition of the transport term $u \omega_x$, and it is the same as (\ref{eq:boussinesq_system}) (when $\theta = 0$) except for the $u_x \omega$ term. The system (\ref{eq:boussinesq_system}) is similar to 2D Euler when $\theta=0$ and is regular in that case. In vorticity form the De Gregorio  equation is given by 
\begin{equation}
	\omega_t + u \omega_x = u_x \omega, \quad u_x = H \omega\,,
	\label{eq:degregorio}
\end{equation}
on $S^1$. The question of blowup has been open since 1990 \cite{gregorio1990}. 

\textbf{We have developed numerical and theoretical evidence that suggests solutions to (\ref{eq:degregorio}) \textit{do not} blow up}.  Given the periodicity of the problem, I used a numerical method with spectral convergence to gain intuition about the behavior of solutions before we began working on the theory. Numerical simulation of this equation has been done previously, but we have observed some new features, such as convergence to equilibria as $t \to \infty$. Using the numerical data as our guide, we have managed to prove local linear stability of the manifold of equilibria $A \sin x + B \cos x$, and we are preparing a paper  for publication.

This regularity might be due to some underlying geometric structure. Indeed, (\ref{eq:degregorio}) has many connections to geometry. The equation says that $\omega$ is transported as a vector field. It is also interesting to note that  changing the coefficient of $u_x \omega$ to $2$ from $-1$ gives the equation for geodesic flow on $\textrm{Diff}(S^1)$ with an appropriate right-invariant Riemannian metric. In contrast to solutions of our \textit{original} equation, the authors of \cite{preston2015} showed that solutions to this \textit{modified} equation blowup generically. 

Geometrically, the addition of the transport term to (\ref{eq:constantin}) allows us to rewrite our equation in terms of the Lie bracket as 
\[
	\omega_t + [u, \omega] = 0.
\]
Solutions evolve by the pushforward of the initial data under a diffeomorphism. This is similar to 3D Euler. The geometric perspective might lead to a deeper understanding of the behavior we observed in the numerical data.
 
Our result of local linear stability for (\ref{eq:degregorio}) is just one step towards a \textit{global} picture. \textbf{My  goal is first to prove local nonlinear stability using perturbation theory and then to prove global regularity for solutions starting from smooth data}. In addition, I hope to analyze the convergence of generic smooth solutions to the manifold of  equilibria of the form $A\sin x+B\cos x$. Some non-generic trajectories can be attracted to unstable manifolds of equilibria of the form $A_m\cos(mx)+B_m\sin(mx)$ for $m \geq 2$.  

Proving global regularity for the De Gregorio equation might provide insight into regularity near the axis of full axi-symmetric 3D Euler with swirl, to which the De Gregorio equation is somewhat connected, as described in \cite{tom2009}


Beyond (\ref{eq:degregorio}), I plan to move from the 1D approximation given in (\ref{eq:boussinesq_system}) and study the regularity of the full 2D Boussinesq flow. While regularity is known for nonzero viscosity and dissipation, the case of zero viscosity and dissipation remains open \cite{hu2016}. 

\textbf{Broader Impact.} In addition to traditional ways of sharing  mathematical research, I have promoted my work via two novel channels. First, I have arranged to give a talk at \textbf{Caf� Scientifique} about fluid dynamics. As mentioned in my personal statement, Caf� Scientifique is an international program that hosts scientists in local venues for research talks accessible to the general public. I will be the first mathematician giving a presentation and am working hard to make my talk engaging and informative. I plan to continue presenting over the next year at Caf� Scientifique and have already pitched several ideas for future talks.

Second, \textbf{I have written an interactive 2D fluid simulator} in JavaScript and WebGL (a graphics library for the web) \cite{stewart2016} and posted it to my blog. My blog is aimed at a wide audience, with articles from trigonometry to detailed results in Lie theory. I hope that posting the interactive fluid solver  will spark the interest of my visitors. If they want to explore further, I have made the code from my work with Professor Sverak freely available. 

\footnotesize
\renewcommand{\section}[2]{}
% clever trick to disable title from: http://tex.stackexchange.com/questions/22645/hiding-the-title-of-the-bibliography

\begin{thebibliography}{aa}
    
    \bibitem{preston2015}
	Bauer, Martin, Boris Kolev, and Stephen C. Preston. ``Geometric investigations of a vorticity model equation.'' \textit{Journal of Differential Equations} \textbf{260}.1 (2016): 478-516.
	
	\bibitem{sverak2014}
	Choi, Kyudong, et al. "On the finite-time blowup of a 1D model for the 3D axisymmetric Euler equations." \textit{arXiv preprint} arXiv:1407.4776 (2014).
	
    \bibitem{constantin1985}
    Constantin, Peter, Peter D. Lax, and Andrew Majda. ``A simple one-dimensional model for the three-dimensional vorticity equation.'' \textit{Communications on pure and applied mathematics} \textbf{38}.6 (1985): 715-724.
	
	\bibitem{gregorio1990}
	De Gregorio, Salvatore. ``On a one-dimensional model for the three-dimensional vorticity equation.'' \textit{Journal of Statistical Physics} \textbf{59}.5-6 (1990): 1251-1263.
	
	\bibitem{hu2016}
    Hu, Weiwei, et al. ``Boussinesq equations with zero viscosity or zero diffusivity: a review.'' \textit{Recent Progress in the Theory of the Euler and Navier-Stokes Equations, Part of London Mathematical Society Lecture Note Series}, available from February (2016).
    
	\bibitem{stewart2016}
	Stewart, Samuel. "Real Time Fluid Simulation on the Web." Academic Blog. N.p., 24 Oct. 2016. Web. 25 Oct. 2016. Available at \url{http://samstewart.github.io/blog/2016/10/real-time-fluid-simulation-on-the-web}.
	
	\bibitem{tom2009}
	Thomas Y.; Lei, Zhen ``On the stabilizing effect of convection in three-dimensional incompressible flows.'' \textit{Comm. Pure Appl. Math.} 62 (2009), no. 4, 501?-564.
\end{thebibliography}

% Reference on nonlinear stability techniques: http://depts.washington.edu/bdecon/workshop2012/g_stability.pdf}.
% Constantine's expository article on Euler equations:
% http://people.cs.uchicago.edu/~const/eule.pdf
% Survey of regularity for navier stokes: http://arxiv.org/pdf/math/0703406v1.pdf.
% https://terrytao.wordpress.com/2010/06/07/the-euler-arnold-equation/

\end{document}
