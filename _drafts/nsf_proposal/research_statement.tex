\documentclass[12pt]{article}
\usepackage[english]{babel}
\usepackage[utf8]{inputenc}
\usepackage{url}
\usepackage{fancyhdr}
\usepackage{mathptmx}
\usepackage{amsmath}
\usepackage[letterpaper, portrait, margin=1in,headheight=60.pt]{geometry}

\title{}
\begin{document}

\fontfamily{ptm}
% Outline: 
% Why is the regularity question in Navier Stokes important?
% 	Terry Tao introduction
% 	Step back to simpler case: Euler equations 
% How do we hope to find blowup?
% 	Axis-symmetric flow might lead to blowup
%		Tornado and tea cup example
%	Perhaps we can create blowup at the boundary
% What kind of model do we choose?
%	Need to close the system
%		Look for proper biot-savart law
% Why our model is special?
%	Better model for 3D Euler	
% How did we solve the problem and how will we approach the next problem?
%	Solved it via numerics and theory
% Why do our results matter?
%	 
% Future directions:
%	Can the blowup for our equation be translated into blowup for Euler?
%	
% Questions for Sverak:
% 1. How could we use this model to generate blowup for Euler 3D? [Behavior near the axis)
% 2. If we can rule out such a singularity near the boundary for Navier Stokes, then what is the point of this? (still break down of Newtonian physics. Superfluid)
% 3. Why do we flip the velocity field in the initial configuration? (best configuration for generating blowup)
% 4. Why does he expect a singularity at the center point (you expect shock formation)
% 5. Why don't we care about Burger's equation? (we do. it's a good 1d model. Read Constantine-Lax 2003 paper)
% 6. Is there something special about the 1D equation that we hope will connect the blowup to the 2D equation?
%	No, actually closer to 3D equation.
% 7. Why isn't there a canonical choice?
%	Hard to decide what to model.
% 8. Can we add some kind of physical explanation for our 1D results? For example, why do we expect the first derivative where the vorticity is zero to be conserved?
% 9. Why this no flux condition? [ exactly why Navier stokes doesn't have blowup near the boundary)
% 10. Why is vorticity so important? How is it related to electromagnitism?
% 11. What is the hyperbolic blowup scenario?
% 12. How do we choose these Biot-Savart laws? [ Described later in Sverak's paper ]

% After discussiong with Sverak
% Future goals:
% 1. Global stability for De Gregorio
% 2. Link to axis-symmetric flow in full 3D euler (with swirl) 
% 3. Perhaps explore Boussinque 1D model for full 2D blowup

% Boussinique flow has singularity at the boudnary from axis symmetric case. The boundary is closed, so we can model it with a 1D equation quite accurately.
% Good motivating papers: http://people.cs.uchicago.edu/~const/eule.pdf
% De Gregorio's motivating paper
% Constantine-Lax-Majda 1D model (2003). De Gregorio is essentially a regularization of this
% w_t + u w_x = 0 regular transport
% 1d Boussinique flow (has blowup?)
% De Gregorio (vorticity transported as a vector, not as a scalar)

% Vorticity transported as scalar for 2D (hence regularity) while vorticity is transported as vector in 3D. From this perspective, De Gregorio tells us 3D flow might be regular.

% the addition of teh transport term w_x u seems to add some kind of regularization effect.

% we can prove rigorously our local stability, and have numerically global attractor. There is a gap here and I plan to close it	
% what we have
%	Linear stability
% what we don't have
% 	nonlinear stability
%	global attractor (not even preventing blowup)

% what is Euler equations with flow?
Blowup of solutions from smooth initial data for Navier-Stokes is quite literally ``a million-dollar question'' according to the Clay Institute [cite]. But the mathematical fluid dynamicist Constantin thinks that constructing a blowup solution to the Euler equations, instead of Navier-Stokes equation,  is more tractable. He argues that ``the occurrence of an infinite momentum...with positive kinematic viscosity has no physical justification'' [cite]. Despite the lack of viscosity, a blowup solution to Euler equations would already be evidence that the current models are insufficient.

One potential approach, which dates at least back to Burgers [cite], is to study 1D model equations. A simple beginning is a 1D model for 2D Euler studied by Constantin et el [cite]
\[
	\label{eq:constantin}
	\omega_t = u_x \omega
\]
with the Biot-Savart law $u_x = H \omega$. The authors exhibited solutions that blowup in finite time. A more complicated example is a model of 2D Boussinesq flow at the boundary. The 2D Boussinesq system represents a flow in the half-plane that is similar to axi-symmetric 3D Euler. Strong numerical evidence points to a singularity near the boundary, but a rigorous proof is difficult [cite]. A possible 1D model of the boundary behavior, studied in [cite Sverak's paper], is given by
\begin{equation*}
	\label{eq:boussinisq_system}
	\begin{aligned}
		\omega_t + u \omega_x &= \theta_x \cr
		\theta_t + u \theta_x &= 0
	\end{aligned}
\end{equation*}
with the same Biot-Savart law $u_x = H \omega$ as before. First numerical evidence pointed to singularity formation [cite], before Sverak et el exhibited actual blowup solutions. This 1D system models flow at the boundary of the 2D Boussinesq system, which itself is related to axi-symmetric 3D Euler, this 1D model can provide insight into the 3D setting.

System (\ref{eq:constantin}) models vorticity transported as a \textit{scalar}. If we add the transport term $\omega_x u$ then we obtain an equation of De Gregorio [3] given by
\[
	\omega_t + u \omega_x = u_x \omega, \quad u_x = H \omega
\]
on the 1D torus. The question of blowup has been open since 1990 [cite]. When written with the Lie bracket, 
\[
	\omega_t + [u, \omega] = 0,
\]
one can see that $\omega$ is now transported as a one dimensional \textit{vector}. This distinction is similar to the difference between 2D and 3D Euler, and thus the resulting differences in regularity. Solutions to (\ref{eq:constantin}) blow up in finite time, but perhaps the addition of the convection term $u \omega_x$ somehow regularizes the solutions.

Indeed, Professor Sverak, Professor Hao Jia, and I have numerical evidence and theoretical insights that suggest the solutions to (\ref{eq:de_gregorio}) do not blow up. In some sense, adding the convection term to (\ref{eq:majda_model}) regularizes the solution. I wrote a solver with the spectral method to gain intuition about the behavior of solutions. From the numerics, it appears that there are two global attractors: $\sin x$ and $\cos x$. Thus far, we have managed to prove local linear stability of these equilibria and are preparing our paper for publication. Already the results are exciting since we are one step closer to eliminating blowup solutions. 

This project has sparked my interest in 1D model equations for the Euler equations. My goal is first to understand De Gregorio's equation completely. The question of global stability of the equilibrium solutions remains open, though the numerics strongly suggest an affirmative answer. To prove local stability, we have observed that solutions to the linearized equation are complex analytic, allowing us to complexify the problem and apply complex ODE theory to rule out discrete spectra, and hence periodic solutions. Moving to nonlinear stability is more difficult, but a perturbative approach might work. (ask Vladimir about global stability, that is, proving they are attractors. Should I mention blowup results?).

There is a connection between De Gregorio's equation and behavior of axi-symmetric 3D Euler with swirl away from the axis. This might hint that 3D Euler behaves nicely away from the axis. Like the existence of blowup solutions to the 1D model (\ref{eq:boussinisq_system}) is a potential starting point to constructing blowup solutions for 2D Euler, the lack of blowup solutions to the 1D De Gregorio model suggests that axi-symmetric 3D Euler is regular away from the axis. I plan to explore this connection more rigorously.

In the end, we hope to hope to show that De Gregorio's model has no blowup solutions. However, as discussed earlier, the 1D model for 2D Boussinesq flow studied by Sverak et el., exhibited blowup solutions, suggesting a potential singularity in two dimensions. I plan to explore the 1D models for Boussinesq flow, first numerically and then analytically, in the hopes of making a stronger connection between the model equations and the behavior of the full 2D flow.


\textbf{Broader Impact}
We interact with fluids daily but it is easy to overlook their beautiful complexity. Wanting to share my excitement, I have found two novel ways to impact the general public. First, I have arranged to give a talk at Cafe Scientifique about fluid dynamics. Cafe Scientifique is a nation-wide program that hosts scientists for outreach to the general public. The local chapter in Minneapolis is hosted in Bryant Lake-Bowl Theatre, a local bar and bowling alley. I will be the first mathematician giving a presentation. I am working hard to make my presentation accessible and entertaining to nonspecialists by collaborating with a fellow graduate student who studied theatre in college. 

Wanting to impact audiences beyond Minneapolis, I have also written a solver for 2D Navier-Stokes in JavaScript and WebGL and posted it on my blog. Due to recent advances in graphics cards, realtime fluid simulation in the web browser is now possible. The simulator allows the user to play with a two dimensional viscous fluid directly in their web browser, without the need for additional plugins. Even six years ago this would have been impossible. 

\textbf{Intellectual Merit}
Having taken a year long PDE course, a year long fluids course, and passing all the preliminary exams, I am prepared to tackle this project. This semester I am taking a dynamical systems course, a harmonic analysis course, and have am reading functional analysis and spectral theory with a graduate student at University of Illinois. I have begun learning about numerical methods for fluids by writing a web based fluid solver. 

\begin{enumerate}
	\item Constantins paper on fluid flows. Available at \url{http://arxiv.org/pdf/1312.4913v1.pdf}.
	\item 1D model for Boussinesq flow. Available at \url{http://arxiv.org/pdf/1312.4913v1.pdf} 
	\item De Gregorios paper
	\item Sveraks paper in 2014 for double exponential growth (Tao likes this)
	\item Sveraks paper for finite time blowup. http://arxiv.org/pdf/1407.4776.  
\end{enumerate}
\end{document}
