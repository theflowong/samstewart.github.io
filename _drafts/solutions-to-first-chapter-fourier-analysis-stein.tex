%23: 7, Munkres \textsection 24: 1, 3, 10e
% v0.04 by Eric J. Malm, 10 Mar 2005
\documentclass[12pt,letterpaper,boxed]{jhwhw}

\author{Sam Stewart}
\title{Introduction to Fourier Analysis Summer Study}
% set 1-inch margins in the document
\usepackage{enumerate}
\usepackage{amsthm}
\usepackage{amsmath}
\usepackage{amssymb}
\usepackage{marginnote}
\usepackage{float}


\newtheorem{theorem}{Theorem}[section]
\newtheorem{lemma}[theorem]{Lemma}
\newtheorem{proposition}[theorem]{Proposition}
\newtheorem{corollary}[theorem]{Corollary}
\newtheorem{definition}{Definition}

% \newcommand*\conj[1]{\bar{#1}}
\newcommand{\topologyn}{{\mathcal{T}}}
\newcommand{\indic}[1]{{\textbf{1}_{#1}}}
\newcommand{\topologyp}{{\mathcal{T}^\prime}}
\newcommand{\basis}{{\mathcal{B}}}
\newcommand{\topology}[1]{{\mathcal{T}_{#1}}}
\newcommand{\indicator}[1]{{\textbf{1}_{#1}}}
\newcommand{\Zplus}{{\mathbb{Z}_+}}
\newcommand{\Z}{{\mathbb{Z}}}
\newcommand{\Mf}{{\mathcal{M}_F}}
\newcommand{\M}{{\mathcal{M}}}
\newcommand{\eps}{{\varepsilon}}
\newcommand{\lebg}[1]{{\mu_L\left(#1\right)}}
\newcommand{\outerm}[1]{{\mu^*\left(#1\right)}}
\newcommand{\lebm}[1]{{\mu_L\left(#1\right)}}
\newcommand{\measure}[1]{{\mu\left(#1\right)}}
\newcommand{\ring}{\mathcal{R}}
\newcommand{\C}{{\mathbb{C}}}
\newcommand{\K}{{\mathbb{K}}}
\newcommand{\Nn}{{\mathbb{N}}}
\newcommand{\Rplus}{{\mathbb{R}_+}}
\newcommand{\rl}{{\mathbb{R}_l}}
\newcommand{\linfty}{l^\infty}
\newcommand{\closure}[1]{{\text{Cl}\left( #1 \right)}}
\newcommand{\Rmin}{{\mathbb{R}_-}}
\newcommand{\Q}{{\mathbb{Q}}}
\newcommand{\R}{{\mathbb{R}}}
\newcommand{\eR}{{\overline{\mathbb{R}}}}
\newcommand{\lebRing}{{\ring_{\text{Leb}}}}
\newcommand{\Rtwo}{{\mathbb{R}^2}}
\newcommand{\Rn}{{\mathbb{R}^n}}
\newcommand{\norm}[1]{\lVert #1 \rVert}
\newcommand{\dotp}[2]{{#1 \cdot #2}}
\newcommand{\abs}[1]{\left| #1 \right|}
\newcommand{\powerset}[1]{{\mathcal{P}(#1)}}
\newcommand{\puncplane}{{\ring^2 - \{ 0 \} }}
\newcommand{\puncplanen}{{\ring^n - \{ 0 \} }}
\newcommand{\til}[1]{{\widetilde{#1}}}
\newcommand{\degree}[2]{{deg_{#1} (#2)}}
\newcommand{\conj}[1]{{\overline{#1}}}
\newcommand{\series}[1]{{\sum_{k = 1}^\infty {#1}_k}}
\newcommand{\seq}[1]{{\left( {#1} \right)_{n = 1}^\infty }}
\newcommand{\maxm}[2]{{\max \, \left\{ {#1}, \, {#2} \right\} }}
\newcommand{\minm}[2]{{\min \, \left\{ {#1}, \, {#2} \right\} }}
\newcommand{\shortseq}[1]{{\left( {#1} \right) }}
\newcommand{\interior}[1]{{\textrm{Int} \left( {#1} \right)}}
\newcommand{\innerp}[2]{{\left< #1,\, #2\right>}}
% \newtheorem{lemma}[section]{Lemma}

\usepackage{graphicx}
\usepackage{float}

% Note: for other writers, please take a look at the shortcuts I have already defined above.

% TODO: employ roman numerals in the 
\begin{document}
\problem{1}
If $z = x + iy$ is a complex number with $x, y \in \R$, we define
\[
	\abs{z} = (x^2 + y^2)^{1/2}
\]
and call this quantity the modulus or absolute value of $z$.

\begin{enumerate}
	\item What is the geometric interpretation of $\abs{z}$?
	\item Show that if $\abs{z} = 0$, then $z = 0$.
	\item Show that if $\lambda \in \R$, then $\abs{\lambda z} = \abs{\lambda} \abs{z}$, where $\abs{\lambda}$ denotes the standard absolute value of a real number.
	\item If $z_1$ and $z_2$ are two complex numbers, prove that
		\[
			\abs{z_1 z_2} = \abs{z_1}\abs{z_2} \textrm{ and } \abs{z_1 + z_2} \leq \abs{z_1} + \abs{z_2}
		\].
	\item Show that if $z \neq 0$, then $\abs{1/z} = 1/\abs{z}$.
\end{enumerate}

\solution
\begin{enumerate}
	\item The quantity $\abs{z}$ represents the distance in the complex plane from $0 + i0$ to $x + iy$.
	\item Assume that $\abs{z} = 0$. Then $z \conj{z} = 0$ so by the fact that $\C$ is a field, we have $z = 0$ or $\conj{z} = 0$. If $\conj{z} = 0$ then since conjugation is an order two automorphism, we have $z = 0$.
	\item Let $\lambda \in \R$ and $z \in \C$. Then
	\begin{align*}
		\abs{\lambda z} &= \sqrt{\lambda z \conj{\lambda z}} \cr
				&= \sqrt{\lambda z \conj{\lambda} \conj{z}} \cr
				&= \sqrt{\lambda z \lambda \conj{z}} \cr
				&= \sqrt{\lambda^2 z \conj{z}} \cr
				&= \abs{\lambda} \abs{z}.
	\end{align*}
	\item Let $z_1, z_2 \in \C$. Then we have
		\begin{align*}
			\abs{z_1 z_2} &= \sqrt{z_1 z_2 \conj{z_1 z_2}} \cr
				      &= \sqrt{z_1 z_2 \conj{z_1} \conj{z_2}} \cr
				      &= \sqrt{z_1 \conj{z_1} z_2 \conj{z_2}} \cr
				      &= \sqrt{z_1 \conj{z_1}} \sqrt{z_2 \conj{z_2}} \cr
				      &= \abs{z_1} \abs{z_2}.
		\end{align*}

	\item By Problem 2, we have that $\abs{z} = \sqrt{z \conj{z}}$. Using the fact that conjugation is a field automorphism, we have
		\[
			\abs{1/z} = \sqrt{\frac{1}{z} \frac{1}{\conj{z}}} = \frac{1}{\abs{z}}
		\].
\end{enumerate}

\problem{2}
If $z = x + iy$ is a complex number with $x, y \in \R$, we define the complex conjugate of $z$ by
\[
	\conj{z} = x - iy.
\]

\begin{enumerate}
	\item What is the geometric interpretation of $\conj{z}$?
	\item Show that $\abs{z}^2 = z \conj{z}$.
	\item Prove that if $z$ belongs to the unit circle, then $1/z = \conj{z}$.
\end{enumerate}

\solution
\begin{enumerate}
	\item Conjugation reflects the complex number about the real axis in the complex plane.
	\item Computation shows that
		\[
			z \conj{z} = (x + iy)(x - iy) = x^2 + y^2 = \abs{z}^2.
		\]
	\item Assume that $\abs{z} = 1$. Then
		\[
			\frac{1}{z} = \frac{\conj{z}}{z \conj{z}} = \frac{\conj{z}}{\abs{z}^2} = \conj{z}.
		\]
\end{enumerate}

\problem{4}
For $z \in \C$, we define the complex exponential by
\[
	e^z = \sum_{n = 0}^\infty \frac{z^n}{n!}.
\]

\begin{enumerate}
	\item Prove that the above definition makes sense, by showing that the series converges for every complex number $z$. Moreover, show that the convergence is uniform on every bounded subset of $\C$.
	\item If $z_1, z_2$ are two complex numbers, prove that $e^{z_1} e^{z_2} = e^{z_1 + z_2}$.
	\item Show that if $z$ is purely imaginary, that is, $z = iy$ with $y \in \R$, then
		\[
			e^{iy} = \cos y + i \, \sin y.
		\]
	      This is Euler's identity.
	\item More generally,
		\[
			e^{x + iy} = e^x (\cos y + i \sin y)
		\]
	      whenever $x, y \in \R$, and show that
	      \[
		      \abs{e^{x + iy}} = e^x.
	      \]
\end{enumerate}

\solution
\begin{enumerate}
	\item We first show pointwise convergence. By properties of the norm, we have
		\[
			\abs{\frac{z^n}{n!}} = \frac{\abs{z}^n}{n!}.
		\]
		The Ratio Test shows that
		\[
			\sum_{n = 0}^\infty \frac{\abs{z}^n}{n!} < \infty,
		\]
		so by Problem 3c we have that
		\[
			\sum_{n = 0}^\infty \frac{z^n}{n!}
		\]
		converges. 

		For uniform convergence, we apply the Weierstrass M-Test. Let $S$ be a bounded subset of $\C$. Then for any  $z \in S$, we have $\abs{z} \leq C \in \R$. Hence,
		\[
			\abs{\frac{z^n}{n!}} \leq \frac{C^n}{n!}.
		\]
		Again by the Ratio Test, we see the series
		\[
			\sum_{n = 0}^\infty \frac{C^n}{n!}
		\]
		converges. By the Weierstrass M-Test, this implies that
		\[
			\sum_{n = 0}^\infty \frac{z^n}{n!}
		\]
		converges absolutely and uniformly on $S$.
	\item Let $z = iy$ with $y \in \R$. Then by definition
		\begin{align*}
			e^{iy} &= \sum_{n = 0}^\infty \frac{(iy)^n}{n!} \cr
			&= \sum_{n \text{ even}} \frac{(-1)^{n/2} y^n}{n!} + \sum_{n \text{ odd}} \frac{(-1)^{(n - 1)/2} i y^n}{n!} \cr
			&= \cos(y) + i \sum_{n \text{ odd}} \frac{(-1)^{(n - 1)/2} i y^n}{n!} \cr
			&= \cos(y) + i \sin(y).
		\end{align*}
	\item By Problem 4b, we have that
		\[
			e^{x + iy} = e^x e^{iy}.
		\]
	      Euler's identity then gives us
	      \[
		      e^{x + iy} = e^x (\cos(y) + i \sin(y)).
	      \]
	      Problem 1d shows that
	      \[
		      \abs{e^{x + iy}} = \abs{e^x (\cos(y) + i \sin(y))} = \abs{e^x} \abs{(\cos(y) + i \sin(y))} = e^x \cdot 1 = e^x.
	      \]
\end{enumerate}

\problem{10}
We look for a solution of the steady-state heat equation $\Delta u = 0$ in the rectangle $R = \{ (x, y) \mid 0 \leq x \leq \pi, 0 \leq y \leq 1 \}$ that vanishes on the vertical sides of $R$, and so that
\[
	u(x, 0) = f_0(x) \text{ and } u(x, 1) = f_1(x),
\]
where $f_0$ and $f_1$ are initial data which fix the temperature distribution on the horizontal sides of the rectangle.

Use separation of variables to show that if $f_0$ and $f_1$ have Fourier expansions
\[
	f_0(x) = \sum_{k = 1}^\infty A_k \sin kx \text{ and } f_1 = \sum_{k = 1}^\infty B_k \sin(kx)
\]
then
\[
	u(x, y) = \sum_{k = 1}^\infty \left( \frac{\sinh k(1 - y)}{\sinh k} A_k + \frac{\sinh ky }{\sinh k} B_k \right) \sin kx.
\]

\solution
Assume that $u = F(x) G(y)$ and plug this into $\Delta u = 0$. We get
\[
	G(y) F''(x) + F(x) G''(y) = 0.
\]
Separation of variables then gives
\[
	\frac{-F''(x)}{F(x)} = \frac{G''(y)}{G(y)}.
\]
Both sides must be equal to a constant $\lambda$ because they depend on different variables so we obtain the following system
\[
	\begin{cases}
		F'' + \lambda F &= 0 \cr
		G'' - \lambda G &= 0
	\end{cases}
\]
Due to the boundary conditions in $x$, we want $F(x)$ to be periodic, so we rule out the case $\lambda < 0$. Assume that $\lambda = m^2$. We have the following two solutions
\[
	\begin{aligned}
		F(x) &= A \cos mx + B \sin mx \cr
		G(y) &= \frac{C_2 e^{my} - C_1 e^{-my}}{m}
	\end{aligned}
\]
The boundary conditions $u(0, y) = u(\pi, y) = 0$ forces $A = 0$. If $B \neq 0$, then $\sin mx = 0$ so $m$ must be an integer.

Now we use the two remaining boundary conditions to determine $C_1$ and $C_2$ (after absorbing the constant $B$). We have
\[
	\begin{aligned}
		u(x, 0) &=  \sin mx \left( \frac{C_2  - C_1 }{m} \right) = f_0 = \sum_{m = 1}^\infty A_m \sin(m x) \cr
		u(x, 1) &=  \sin mx \left( \frac{C_2 e^{m} - C_1 e^{-m}}{m} \right) = f_1 = \sum_{k = 1}^\infty B_m \sin mx.
	\end{aligned}
\]
Using the uniqueness of Fourier coefficients (or just multiplying both sides by $\sin mx$ and integrating), gives
\[
	\begin{aligned}
		\frac{C_2 e^m - C_1 e^{-m}}{m} &= B_m \cr
		\frac{C_2 - C_1}{m} &= A_m.
	\end{aligned}
\]
Solving for $C_1$ and $C_2$ gives the following (slight nasty) expressions
\[
	\begin{aligned}
		C_1 &= \frac{\left( B_m - A_m e^m \right) m}{2 \sinh m} \cr
		C_2 &= \frac{A_m m (2 \sinh(m) - e^m) + B_m}{2 \sinh m}.
	\end{aligned}
\]
Summing and combining the coefficients of $A_m$ and $B_m$ gives, after some algebra,
\[
	\begin{aligned}
		G(y) &= \frac{C_2 e^{my} - C_1 e^{-my}}{m} \cr
		     &= \frac{\sinh m(1 - y)}{\sinh m} A_m + \frac{\sinh{my}}{\sinh{m}} B_m.
	\end{aligned}
\]
The final solution is thus
\[
	u(x, y) = \sum_{m = 1}^\infty \sin mx \left( \frac{\sinh m(1 - y)}{\sinh m} A_m + \frac{\sinh{my}}{\sinh{m}} B_m \right).
\]

\problem{10}
Show that the expression of the Laplacian
\[
	\Delta = \partial_x^2 + \partial_y^2
\]
is given in polar coordinates by the formula
\[
	\Delta = \partial_r^2 + \frac{1}{r} \partial_r + \partial{1}{r^2} \partial_\theta^2.
\]
Also, prove that
\[
	\abs{\partial_x u }^2 + \abs{\partial_y u}^2 = \abs{\partial_r u}^2 + \frac{1}{r^2} \abs{\partial_\theta u}.
\]
\solution
Please see document \url{changing_coordinates_of_differential_operators.tex}.
\end{document}
